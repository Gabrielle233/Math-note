\documentclass{article}
\usepackage{amsthm}
\usepackage{amssymb}
\usepackage[utf8]{inputenc}
\usepackage[english]{babel}
\newtheorem*{theorem}{Theorem}
\newtheorem*{definition}{Definition}
\newtheorem*{proposition}{Proposition}

\title{Notes on Baire Category Theorem}
\author{Gabrielle Li}
\date{April, 2020}

\begin{document}

\maketitle

\begin{definition}[Nowhere dense]
Let X be a topological space. A subset A is called nowhere dense in X if the interior of the closure of A is empty, i.e. Int ($\overline{A}) = \emptyset$.
\end{definition}

\begin{proposition} Let X be a topological space, then
\begin{enumerate}
    \item Any subset of a nowhere dense set is nowhere dense.
    \item The union of finitely many nowhere dense sets is nowhere dense.
    \item A subset is nowhere dense iff its closure is nowhere dense.
    \item A subset is nowhere dense iff its complement contains a dense set.
\end{enumerate}
\end{proposition}

\begin{definition}[Meagre set/First category]
In a topological space X, a subset S is a meagre subset (or of the first category) if it is the union of countably many nowhere dense subsets in X.
\end{definition}

\begin{definition}[Second category]
In a topological space X, a subset S is of the second category if it is not of first category.
\end{definition}

\begin{proposition} Let X be a topological space, then
\begin{enumerate}
    \item Any subset of a meager set is meager.
    \item The union of finitely many meagre sets is nowhere meagre.
\end{enumerate}
\end{proposition}

\begin{theorem}[Baire Category Theorem] Let X be a complete metric space, then
\begin{enumerate}
\item Let $G_{1}, G_{2}$...be a sequence of dense open subsets X. Then $G = \bigcap^{\infty}_{n=1}G_{n}$ is dense in X.
\item If X is the union of countably many closed sets, then at least one of the closed sets has non-empty interior. This means that a complete metric space is of the second category.
\item The complement of a meagre subset is dense and of the second category.
\end{enumerate}
\end{theorem}

\begin{theorem}[Principle of uniform boundness (Osgood)] Let U be a set of the second category in a metric space X and let $\mathcal{F}$ be a family of continuous functions $f: X \rightarrow \mathbb{R}$ such that {$f(u): f \in \mathcal{F}$} is bounded for every u $\in$ U. Then the elements of $\mathcal{F}$ are uniformly bounded for in some ball B in X, i.e. $\vert f(x) \vert$ is bounded by some n for all $f \in \mathcal{F}$ and all $x \in B$.
\end{theorem}

\end{document}
